\chapter{Überblick und Workflow}
Dieses Programm wird verwendet, um aus Spritesheets 2D-Animationen zu erstellen und diese zu speichern, sodass sie dann in anderen Anwendungen wie z.B. einem Videospiel verwendet werden können. Das Programm zeigt dazu ein GUI an, mit dem zuerst ein Spritesheet im PNG-Format geöffnet wird. Anhand der Größe des Spritesheets wird die Größe der einzelnen enthaltenen Sprites geschätzt, diese kann dann danach aber noch manuell vom User angepasst werden. Außerdem sind zwei Checkboxen vorhanden: "Preview animation" steuert die Anzeige einer Vorschau für die aktuell ausgewählte Animation; "Lines between sprites" aktiviert die Darstellung von Linien, die die Grenzen zwischen den einzelnen Sprites markieren und so die durch das Input-Feld "Sprite dimensions" festgelegte Größe veranschaulichen.

Darunter befindet sich der Teil des Interfaces, in dem Animationen hinzugefügt und bearbeitet werden können. Über eine Combobox kann eine Animation ausgewählt werden, mit den Buttons "Add" und "Remove" werden Animationen hinzugefügt oder entfernt. "Set name" öffnet ein Popup, in dem ein neuer Name für die aktuelle Animation eingegeben werden kann.

Ist eine Animation ausgewählt, werden darunter die einzelnen Schritte der Animation angezeigt. Neue Schritte können per "Add step" hinzugefügt, vorhandene mit dem entsprechenden "Remove"-Button entfernt werden. Zu jedem dieser Schritte werden zwei Eingabefelder angezeigt: "Sprite id" und "Duration". Ersteres bestimmt den Index des Sprites aus dem Spritesheet, das andere die Dauer des Schrittes. Eine Duration von \lstinline{1.00} entspricht hier einem Frame bei 60 Frames pro Sekunde.

Wurde eine Animation erstellt, kann diese per "Save" bzw. "Save as..." gespeichert werden. Dabei wird die Dateiendung \lstinline{.anim} verwendet. Es handelt sich bei dem Format um eine Textdatei, die von diesem Programm auch wieder geöffnet und bearbeitet werden kann.